\documentclass{agenda}
\usepackage{hyperref}

\begin{document}
\group{CS Women Co-Chairs}
\objective{
  Delegate responsibilities for upcoming events, decide role of the lunches, discuss formal documentation.
}

\invited{
  Cassian Corey (cjcorey@umass),
  Myung-ha Jang (mhjang@cs), 
  Alexandra Meliou (ameli@cs), 
  Emma Tosch (etosch@cs), 
  Niha Venkatathri (nvenkata@umass)
}

\attended{
  Alexandra Meliou (ameli@cs), 
  Emma Tosch (etosch@cs), 
}

\scribe{
  Emma Tosch
}

\dateproposed{January 8, 2015}
\dateoccured{January 8, 2015}

%\agendaheader
\minutesheader

\begin{agendaitem}{0}{Come prepared with a laptop and make sure you have access to the CS Women Trello boards.}
  I also recommend either printing out this agenda or pulling the source off github to take notes.
\end{agendaitem}

\begin{agendaitem}{20}{Introductions, post-mortem on last semester.}
  \begin{enumerate}
  \item What should be done differently? 
    \minutesitem{
      Alexandra emphasized the need for organization, especially
      in event planning and other activities that are
      outward-facing. A solution we came up with was to have
      co-chair training and resources on event planning. I can
      run an ``Event Planning 101'' if needed.
    }
  \item What were our objectives? Did we reach them? 
  \item \actionitem{Identify one small, easily accomplished
    improvement for this semester and post this on the CSWomen
    Trello.} 
    \minutesitem{
      Alexandra: Famous women highlights (posted on the website or by
      email). The focus will be on finding role models and sending a
      positive message.
    }
  \item \actionitem{Identify one large goal for this semester and
    write a blog post about this.} 
    \minutesitem{
      Alexandra: Plan a mini-conference for UMass, the 5-colleges, and
      maybe some other institutions. This could be funded by NCWIT
      money. 
    }
  \end{enumerate} 
\end{agendaitem}

\begin{agendaitem}{15}{Delegate responsibilities for upcoming events.}
    Proposed events can be found on the \href{http://umasscswomen.weebly.com/events.html}{\underline{CS Women's Events Page}}. Details can be found on the Trello board, where most of this discussion should take place. I've included here the most time-sensitive or items in greatest need of group discussion. 
    
    \paragraph{Clothing Swap.} This should be done at the start of the
    semester (i.e., two weeks or so). We will need minimal resources
    and the event should cost us little to nothing. We will need to
    \actionitem{agree to host this event}, \actionitem{book a room},
    and \actionitem{draft a description of the event and an
      invitation}. \emph{Update (01/07/15): We've booked 150/151 from
      5-8pm on 1/22/15}
    \minutesitem{
      Alexandra and I agreed that this event would best be
      within-group for this semester.
    }
    
    \paragraph{Chinese New Year} occurs on Feb. 19th. We will need
    someone to \actionitem{take responsibility for this event and
      brainstorm activities}. 
    \minutesitem{
      Will call for participants/planners in the email. This event
      should be open to the entire department and should be ``family friendly.''
    }
     
    \paragraph{Egg Drop Contest.} I think this event is best suited
    for later in the semester. However, we will need to do quite a bit
    of preparation. We should email the group now, asking for
    \actionitem{materials ``donations.''}
    \minutesitem{
      We will need to determine a location, then a date. Rooftops are
      out. Alexandra suggested contacting corporate partners to donate
      prizes for the winner.
    }
       Regardless of what we decide to do regarding events, we will
       need to \actionitem{email the CS Women group a link to the
         events page} so that anyone who has ideas for events or wants
       to get involved knows how. 
\end{agendaitem}

\begin{agendaitem}{15}{Decide what to do about the lunches.}
    Do we still want to hold lunches?
    \begin{enumerate}
        \item {\bf Yes.} We will need to \actionitem{book the room(s) now} and sketch a preliminary 
        plan for the topic of each meeting. \emph{Update (01/07/15): We've booked 151 for the lunches; see the calendar for the exact dates.}
        \item {\bf No.} Discuss:
            \begin{itemize}
                \item What is the goal of the lunches?
                \item Are the lunches deficient in some way?
                \item The department pays for the lunches; will we ``lose'' that money if we don't use it?
            \end{itemize}
    \end{enumerate}
    \minutesitem{
      We mostly discussed ways to make the meetings more friendly,
      open to new participants, etc. Alexandra suggested a kind of
      musical chairs. I am going to reach out to some of the newer
      students to get to know them better, to help with moderating
      discussions. 

      I booked lunch dates with Leeanne after she emailed us. We will 
    }
\end{agendaitem}

\begin{agendaitem}{10}{Discuss Formal Documentation.}
    We should so some meta-planning. The group does not have a
    charter, nor are the roles documented. This should be a
    preliminary discussion that will be continued throughout the
    semester. Those interested in working on documentation should meet
    separately to draft the documents, which will later need to be
    ratified by the group.
    \minutesitem{
      Punted.
    }
\end{agendaitem}

\begin{agendaitem}{15}{Department Stats, Outreach, etc.}
  We still don't know who our members are or should be. Discuss ways to
  find out how many students we should be reaching vs. how many we are.
  \minutesitem{ 
    Students who have marked ``F'' on forms asking for sex are added to the mailing list via a script that Leeanne
    runs. All others are manually added.

    Alexandra suggests eamiling grads and undergrads at the start of
    the semester to find out if anyone wants to be on the list who
    isn't.

    Niha and others have brought up that many undergraduates aren't on
    our radar because they don't declare the major until later. Send
    co-chairs to early classes to advertise the group. Make sure that
    students who may not personally identify as ``CS'' women are
    welcomed (they needn't have joined the major).
  }
\paragraph{WeCode event}

\emph{Update from Niha (01/08/15):
    \begin{enumerate}
        \item All tickets have been purchased. 
        \item Camille from MHC is trying to organize transport for all 5 colleges. Transportation costs will be split up among the five colleges in proportions to the number of students attending from each college.
        \item Leeanne from UMass is helping coordinate transport with Camille.
        \item We still have 5 people on the waitlist. Once transport is confirmed we will know how big our buses/vans are going to be. Depending on the available seats, we will take people off waitlist. Lori has already agreed to fund 5 more tickets.
    \end{enumerate}
    }

\end{agendaitem}

\begin{agendaitem}{5}{Roles and Responsibilities}
  Discuss any specialized roles we need (e.g., webmaster, social
  network maintainer, lunch organizer) and whether they should be
  delegated to co-chairs or additional committees.
  \minutesitem{
    We decided to play this by ear. 
  }
\end{agendaitem}


\end{document}
